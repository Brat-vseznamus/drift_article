% \documentclass[8pt, a4paper, twocolumn]{article}
\documentclass[12pt, a4paper]{article}

\usepackage{amsmath}
\usepackage{amssymb}
\usepackage{amsthm}
\usepackage[utf8]{inputenc}
\usepackage[english]{babel}
\usepackage{indentfirst}
\usepackage{graphicx}
\usepackage{mathtext}
\usepackage[T1,T2A]{fontenc}
\usepackage{listings}
\usepackage{hyperref}
\usepackage{stmaryrd}
\usepackage{a4wide}

\lstset{basicstyle=\ttfamily\footnotesize,breaklines=true}

\title{\text{Additive drift with tail bounds}}

% \date{04-08-2021}

\newcommand\setItemnumber[1]{\setcounter{enumi}{\numexpr#1-1\relax}}

\newcommand{\gfun}{\mathbf{g}}

\newcounter{casenum}
\newenvironment{caseof}{\setcounter{casenum}{1}}{\vskip.5\baselineskip}
\newcommand{\case}[2]{\vskip.5\baselineskip\par\noindent {\bfseries Случай \arabic{casenum}:} #1\\#2\addtocounter{casenum}{1}}

\newtheorem{theorem}{Theorem}[section]
\newtheorem{corollary}{Corollary}[theorem]
\newtheorem{lemma}[theorem]{Lemma}
\theoremstyle{remark}
\newtheorem*{remark}{Remark}

\newcommand{\expx}[1]{e^{-|x|^{#1}}}
\newcommand{\expxpoz}[1]{e^{-x^{#1}}}
\newcommand{\infint}[1]{\int_{-\infty}^{+\infty} #1 \, dx}
\newcommand{\infintpoz}[1]{\int_{0}^{+\infty} #1 \, dx}
\newcommand{\der}[2]{\frac{\partial #1}{\partial #2}}
\newcommand{\cm}{coming soon\dots}

\newcommand{\underrel}[2]{\mathrel{\mathop{#2}\limits_{#1}}}

\author{Vihnin F.
    \and Antipov D.
    \and Sinyachenko N.}


\begin{document}
\maketitle

\section{Introduction}
\cm

\section{Existing tools}
\cm
\subsection{Negative drift}
\begin{theorem}[Drift Theorem for Lower Bounds] \label{thm:neat}
    Let $\{X_t\}_{t \ge 0}$ be a Markow process over a set of states $\mathbb{S}$, and $\mathbf{g} : \mathbb{S} \rightarrow \mathbb{R}$ a function that assigns to every state a non-negative real numbers. Pick two real numbers $a(l)$ and $b(l)$ which depend on a parameter $l \in \mathbb{R}^+$ such that $0 < a(l) < b(l)$ holds and let the random variable $T$ denote
    the earliest point in time $t \ge 0$ where $\mathbf{g}(X_t) \leq a(l)$ holds.
    
    If there are constants $\lambda > 0$ and $D \ge 1$ and a polynomial $p(l)$ taking only positive values, for which the following four conditions hold

    \begin{enumerate}
        \item $\gfun(X_0) \ge b(l)$
        \item $\forall t \ge 0 : E[e^{-\lambda(\gfun(X_{t + 1} - \gfun(X_t)))} |\ X_t,\ \gfun(X_t) < b(l)] \leq 1 - \frac{1}{p(l)} = \rho$
        \item $\forall t \ge 0 : E[e^{-\lambda(\gfun(X_{t + 1} - b(l)))} |\ X_t,\ \gfun(X_t) \ge b(l)] \leq D$
    \end{enumerate}
    then for all time bounds $B \ge 0$, the following upper bound on probability for random variable $T$:

    \begin{equation}
        \Pr[T \leq B] \leq e^{\lambda(a(l) - b(l))} \cdot B \cdot D \cdot p(l)
    \end{equation}
\end{theorem}

To prove this theorem we need to use following lemma

\begin{lemma}[Hajek lemma 2.8] \label{lm:haj}
    If conditions 2 and 3 are met, then

    $$\forall t \ge 0 :\Pr[\gfun(X_t) \leq a(l)] \leq \rho^te^{\lambda(a(l) - \gfun(X_0))} + \frac{1 - \rho^t}{1 - \rho} D e^{\lambda(a(l) - b(l))}$$
    
\end{lemma}

\begin{proof}[Proof of Theorem \ref{thm:neat}]
    
    First of all we need to describe fact that time $T$ is less or equals $B$. It means that at any time $t \leq B$  event $\gfun(X_t) \leq a$ could have happened for first time, then

    % \[
    %     \Pr[T \leq B] = \sum_{k=1}^{B} \Pr[\gfun(X_k) \leq a \land \gfun(X_{k - 1}) > a \land ... \land \gfun(X_0) > a)] 
    % \]
    
    \begin{align*}
        \Pr[T \leq B] &= \sum_{k=1}^{B} \Pr[T = k] \\
        &= \sum_{k=1}^{B} \Pr[\gfun(X_k) \leq a \land \gfun(X_{k - 1}) > a \land ... \land \gfun(X_0) > a)] \\
        & \leq \sum_{k=1}^{B} \Pr[\gfun(X_k) \leq a]
    \end{align*}

        By Lemma $\ref{lm:haj}$:

    \begin{align*}
        \Pr[T \leq B] & \leq \sum_{k=1}^{B} \rho^k e^{\lambda(a(l) - \gfun(X_0))} + \frac{1 - \rho^k}{1 - \rho} D e^{\lambda(a(l) - b(l))}\\
        & = \rho \frac{1 - \rho^B}{1 - \rho} e^{\lambda a(l)}(e^{-\lambda\gfun(X_0)} - \frac{D}{1 - \rho} e^{-\lambda b(l)}) + e^{\lambda(a(l) - b(l))} \frac{B D}{1 - \rho}
    \end{align*}

    Because $\gfun(X_0) \ge b(l)$:

    \begin{align*}
        \Pr[T \leq B] & \leq \rho \frac{1 - \rho^B}{1 - \rho} e^{\lambda (a(l) - b(l))}(1 - \frac{D}{1 - \rho}) + e^{\lambda(a(l) - b(l))} \frac{B D}{1 - \rho}
    \end{align*}
    
    Then because $\frac{D}{1 - \rho} \ge 1 $:
    \begin{align*}
        \Pr[T \leq B] & \leq e^{\lambda(a(l) - b(l))} \frac{B D}{1 - \rho}\\
        & = e^{\lambda(a(l) - b(l))} \cdot B \cdot D \cdot p(l)
    \end{align*}

    Finally

    $$\Pr[T \leq B] \leq e^{\lambda(a(l) - b(l))} \cdot B \cdot D \cdot p(l)$$

\end{proof}
\subsection{Theorem for sub-gaussian processes}
\cm
\subsection{Comparing requirements}
Для использования дрифт теорем, нам необходимо, чтобы выполнялись некоторые ограничения на последовательность случайных величин, например следующее:
$$
\exists \delta \geq 0,\ c \in \mathbb{R} \ \forall  t \geq 0 \ \forall \gamma \in [0, \delta] : $$$$
E[e^{\gamma (X_{t + 1} - X_{t})}] \leq e^{\frac{c \gamma^2}{2}},$$

или

$$\exists \delta \geq 0,\ p \geq 1 \ \forall t \geq 0 : $$$$
E[e^{\gamma (X_{t + 1} - X_{t})}] \leq 1 - \frac{1}{p}\ .$$

Чтобы сравнить строгость требований давайте проанализируем поведение $E[e^{\gamma X}]$ относительно $\gamma$. Так как распределения могут быть любыми, то стоит изучить их по отдельности, и для удобства начнем с непрерывных. 

\subsubsection*{Непрерывный случай}

Вспомним определение матожидание от функции:

\[
E[e^{\gamma X}] = \int_{-\infty}^{+\infty} f(x) e^{\gamma x} \,dx,    
\]

где $f(x)$ - плотность распределения.

\hfill

Рассмотрим первую производную по $\gamma$:

\begin{align*}
\frac{\partial E[e^{\gamma X}]}{\partial \gamma} &= \frac{\partial}{\partial \gamma} \int_{-\infty}^{+\infty} f(x) e^{\gamma x} \, dx = \int_{-\infty}^{+\infty} f(x) \frac{\partial e^{\gamma x}}{\partial \gamma} \, dx = \int_{-\infty}^{+\infty} x f(x) e^{\gamma x} \, dx = \\
&= \int_{-\infty}^{0} x f(x) e^{\gamma x} \, dx + \int_{0}^{+\infty} x f(x) e^{\gamma x} \, dx
\end{align*}

Заметим, что $\forall x \leq 0 : e^{\gamma x} \leq 1$, а для $\forall x \geq 0 : e^{\gamma x} \geq 1 + \gamma x$. В таком случае получается, (так как $\forall x : f(x) \geq 0$): 

\begin{align*}
    \int_{0}^{+\infty} x f(x) e^{\gamma x} \, dx &\geq \int_{0}^{+\infty} x f(x) (1 + \gamma x)\, dx    \\
    \int_{-\infty}^{0} x f(x) e^{\gamma x} \, dx &\geq \int_{-\infty}^{0} x f(x)\, dx    
\end{align*}

В таком случае:

\begin{align*}
\frac{\partial E[e^{\gamma X}]}{\partial \gamma} &\geq  \int_{0}^{+\infty} x f(x) (1 + \gamma x)\, dx + \int_{-\infty}^{0} x f(x)\, dx  \\
&= E[X] + \gamma \int_{0}^{+\infty} x^2 f(x) \, dx
\end{align*}

Назовём $\int_{0}^{+\infty} x^2 f(x) \, dx = c \geq 0$

Тогда если мы проинтегрируем обратно, то получим ограничение на матожидание снизу:

\[
E[e^{\gamma X}] = E[1] + \int_0^\gamma \frac{\partial E[e^{\gamma X}]}{\partial \gamma} \, d\gamma \geq 1 + \gamma E[X] + \gamma^2 \frac{c}{2}    
\]
% Рассмотрим несколько случаев:

% \begin{caseof}
%     \case{$E[X] > 0.$}{

%         В таком случае матожидание экспоненты не может быть ниже 1, а также в окрестности нуля растет быстрее чем любая $e^{\frac{\gamma^2 c }{2}}$, так как: 
        
%         $$\frac{\partial e^{\frac{\gamma^2 c}{2}}}{\partial \gamma} = c\gamma e^{\frac{\gamma^2 c}{2}} = g(\gamma)$$

%         Но $g(0) = 0$, тогда как $E[X] > 0$, следовательно, что в любой окрестности нуля:
        
%         $$\forall c > 0 \, \exists \gamma_c > 0 \, \forall \gamma \in (0, \gamma_c) : e^{\frac{\gamma^2 c}{2}} < E[e^{\gamma X}]$$

%         Следовательно обе границы сверху не выполнимы.
%     }
%     \case{$E[X] < 0$.}{

%         В этом случае, аналогично предыдущему случаю, в некоторой окрестности нуля основной вклад в производную вносит именно слагаемое равно матожиданию, так как:

%         $$\frac{\partial E[e^{\gamma x}]}{\partial \gamma} \Big|_{\gamma = 0} = E[X] < 0$$

%         Но в таком случае выполняются оба ограничения! Так как:

%         $$\exists \gamma_0 > 0 : \forall \gamma \in (0, \gamma_0) : E[e^{\gamma X}] < 1$$

%         Так что доказательство того, что процесс является сабгауссовым, равносильно нахождению $p$ во втором органичении.


%         }
%     \case{$E[X] = 0.$}{

%         Второе ограничение явно не выполянется, так как матожидание экпоненты положительное, то вот точно ли не выполняется ограничение 1 еще предстоит узнать. Этот вопрос будет рассмотрен в $\hyperref[sec:zero]{\text{соотвествующем параграфе}}$.
%     }
% \end{caseof}

\subsubsection*{Дискретный случай}

Аналогично вспомним определение матожидания:

\begin{align*}
    E[e^{\gamma X}] = \sum_{i = 1}^{+\infty} e^{\gamma x_i} Pr[x = x_i],
\end{align*}

где $Pr$ - функция вероятности.

\hfill

Аналогично рассмотрим множество $P = \{i \in \mathbb{N}: x_i \geq 0\}$. Тогда можно переписать: 

\begin{align*}
    E[e^{\gamma X}] = \sum_{i = 1}^{+\infty} e^{\gamma x_i} Pr[x = x_i] = \sum_{i \in P} e^{\gamma x_i} Pr[x = x_i] + \sum_{i \in \mathbb{N}\setminus P} e^{\gamma x_i} Pr[x = x_i],
\end{align*}

В таком случае, аналогично возьмем производную и рассмотрим пот отдельности слагаемые из множества P и остальные:

\begin{align*}
    \der{E[e^{\gamma X}]}{\gamma} &= \sum_{i = 1}^{+\infty} \frac{\partial e^{\gamma x_i}}{\partial \gamma} Pr[x = x_i] = \sum_{i = 1}^{+\infty} x_i e^{\gamma x_i} Pr[x = x_i] \\
    &= \sum_{i \in P} x_i e^{\gamma x_i} Pr[x = x_i] + \sum_{i \in \mathbb{N}\setminus P} x_i e^{\gamma x_i} Pr[x = x_i] \\
    &\geq \sum_{i \in P} x_i (1 + {\gamma x_i}) Pr[x = x_i] + \sum_{i \in \mathbb{N}\setminus P} x_i Pr[x = x_i] \\
    &= E[X] + \gamma\sum_{i \in P} x_i^2 Pr[x = x_i]
\end{align*}

Также обозначим, $c = \sum_{i \in P} x_i^2 Pr[x = x_i] \geq 0$ и тогда ровно также: 

\[
E[e^{\gamma X}] \geq 1 + \gamma E[X] + \gamma^2 \frac{c}{2}  
\]

Так как формулы в обоих случаях получились одинаковые, то дальнейшие рассуждения будут верны для обоих случаев (иначе различия будут уточнены).

\subsubsection{Impracticability for positive drift}

В таком случае матожидание экспоненты не может быть ниже 1, а также в окрестности нуля растет быстрее чем любая $e^{\frac{\gamma^2 c }{2}}$, так как: 
        
$$\frac{\partial e^{\frac{\gamma^2 c}{2}}}{\partial \gamma} = c\gamma e^{\frac{\gamma^2 c}{2}} = g(\gamma)$$

Но $g(0) = 0$, тогда как $E[X] > 0$, следовательно, что в любой окрестности нуля:

$$\forall c > 0 \, \exists \gamma_c > 0 \, \forall \gamma \in (0, \gamma_c) : e^{\frac{\gamma^2 c}{2}} < E[e^{\gamma X}]$$

Следовательно обе границы сверху не выполнимы.

\subsubsection{Zero drfit case}
\label{sec:zero}

Здесь я буду рассматривать случай симметричных функций плотности вероятности, так как их матожидание точно 0. Далее я обобщу на более общий случай.

\paragraph*{Экспоненты от степеней икса}


Давайте рассмотрим функции вида

$$f(x) = c \expx{a},\ a > 0$$

Определим $c$

\begin{align*}
    \infint{f(x)} &= \infint{c\expx{a}} = 1
\end{align*}

\begin{align*}
    \infint{\expx{a}} = 2\infintpoz{\expxpoz{a}}
\end{align*}

Сделаем замену

\begin{align*}
    u &= x^a \\
    du &= dx\,(a x^{a - 1}) = dx\,(a u^{1 - \frac{1}{a}})
\end{align*}

В таком случае

\begin{align*}
    2\infintpoz{\expxpoz{a}} = \frac{2}{a} \infintpoz{e^{-u}u^{\frac{1}{a} - 1}} = \frac{2\Gamma(\frac{1}{a})}{a}
\end{align*}

Откуда делаем вывод, что $c = \frac{a}{2\Gamma(\frac{1}{a})}
$ и 

$$f(x) = \frac{a}{2\Gamma(\frac{1}{a})} \expx{a}$$

Давайте найдем в таком случае n-ый момент (матожидание от $X^n$): 

\begin{align*}
    E[X^n] = \infint{f(x) x^n} = \frac{\Gamma(\frac{n + 1}{a})}{2\Gamma(\frac{1}{a})} (1 + (-1)^n)
\end{align*}

Теперь мы можем рассмотреть матожидание от $e^{\gamma X}$ в качестве ряда.

\begin{align*}
    E[e^{\gamma X}] = E\left[\sum_{n = 0}^{+\infty}\frac{(\gamma X)^n}{n!}\right] &= \sum_{n = 0}^{+\infty}\gamma^n\frac{E[X^n]}{n!} = \sum_{n = 0}^{+\infty}\gamma^{2n}\frac{E[X^{2n}]}{2n!} =\\
    &= \sum_{n = 0}^{+\infty}\frac{(\gamma^2)^{n}}{n!}\left(\frac{E[X^{2n}]n!}{2n!}\right) = \sum_{n = 0}^{+\infty}\frac{(\gamma^2)^{n}}{n!} a_n
\end{align*}

Третье равенство выполняется, так как все моменты нечётной степени равны нулю в силу симметричности функции.

Рассмотрим предел последовательности $\{\sqrt[n]{a_n}\}$:

\begin{align*}
    A = \lim_{n \to +\infty} \sqrt[n]{a_n} &= \lim_{n \to +\infty} \sqrt[n]{\frac{n! E[X^{2n}]}{2n!}} = 
    \lim_{n \to +\infty} \sqrt[n]{
        \frac{
            \left(\frac{n}{e}\right)^n \sqrt{2\pi n} \
            \Gamma\left(\frac{2n + 1}{a}\right)}{
                \left(\frac{2n}{e}\right)^{2n} \sqrt{4\pi n}\ \Gamma\left(\frac{1}{a}\right)}
        }\\
    &= \lim_{n \to +\infty} \frac{n}{e} \frac{e^2}{4n^2} \sqrt[n]{\Gamma\left(\frac{2n + 1}{a}\right)} \simeq \lim_{n \to +\infty}  \frac{e}{4n} \sqrt[n]{\left(\frac{2n + 1}{ae}\right)^{\frac{2n + 1}{a}}} \\
    &= \lim_{n \to +\infty} \frac{e}{4n} \left(\frac{2n}{ae}\right)^{\frac{2}{a}} = \left(\frac{e^{1 - \frac{2}{a}}}{2^{2 - \frac{2}{a}} a^\frac{2}{a}}\right) \lim_{n \to +\infty} n^{\frac{2}{a} - 1}
\end{align*}

Рассмотрим случаи относительно значения а: 
\begin{caseof}
    \case{$a \geq 2$}{

        Тогда $A$ существует и конечно, что значит, что существует такое $C > 0$, такое что 
        $$\forall n \geq 0 : a_n \leq C^n$$

        Следовательно 

        $$E[e^{\gamma X}] \leq e^{\gamma^2 C}$$
    }
    \case{$a < 2$}{

        Предела не существует, а значит ничего утверждать про сабгаусовость нельзя. 
    }
\end{caseof}

Если провести аналогичные вычисления но для последовательности $\{\sqrt[n]{\frac{a_n}{n!}}\}$, то порогом, относительно $a$, будет 1, где только при $a \geq 1$, ряд сходится.

В таком случае остается выяснить, что происходит на отрезке $a \in (1, 2)$. Для этого воспользуемся другим определением сабгаусовости:

$$\exists K > 0 \ \forall p \geq 1: \left(E[X^p]\right)^{\frac{1}{p}} \leq K \sqrt{p}$$ 


Вычислим (будем считать $p\ \vdots\ 2$): 

\begin{align*}
    E[X^p] &= \frac{\Gamma\left(\frac{p + 1}{a}\right)}{\Gamma\left(\frac{1}{a}\right)} \\
    E[X^p]^\frac{1}{p} &\underrel{p \to +\infty} {=} \left(\frac{p}{e a}\right)^{\frac{p}{a} \frac{1}{p}} = \mathcal{O}(p^\frac{1}{a})
\end{align*}

Для $a \in (1, 2)$ очевидно, что асимптотика больше, чем у корня, так что сабгаусовость не наблюдается.

\subsubsection{Processes with negative drift} 

В этом случае, аналогично случаю положительного дрифта, в некоторой окрестности нуля основной вклад в производную вносит именно слагаемое равно матожиданию, так как:

$$\frac{\partial E[e^{\gamma x}]}{\partial \gamma} \Big|_{\gamma = 0} = E[X] < 0$$

Но в таком случае выполняются оба ограничения! Так как:

$$\exists \gamma_0 > 0 : \forall \gamma \in (0, \gamma_0) : E[e^{\gamma X}] < 1$$

Так что доказательство того, что процесс является сабгауссовым, равносильно нахождению $p$ во втором органичении.

\textbf{*РАССУЖДЕНИЯ ПРО П И КАРТИНКИ*}


\subsection*{Заключение}

В случае $\textbf{\text{нулевого}}$ матожидания гарантировать сабгаусовость всегда мы не можем, что было доказано примером серии четных функций $\expx{a}$.

В случае $\textbf{\text{положительного}}$ матожидания оба критерия не выполнимы, а если оно $\textbf{\text{отрицательное}}$, то оба выполнятся.

\section{Tail bounds}
\subsection{Upper bounds}
Lets take a look on process $\{X_t\}$ and its derivative process $\{Y_t\}$, where $Y_t = X_t - \varepsilon t$ and it has negative drift.

Then we want to bounds $\Pr[T_x \leq t_0]$, where $T_x$ - first time $X_t \geq b$. 

\begin{align*}
    \Pr[T_x \leq t_0] \leq \Pr[X_{t_0} \geq b] \leq \Pr[Y_{t_0} \geq b - \varepsilon t_0]
\end{align*}

Then, because $\{Y_t\}$ - process with negative drift, we may use the corresponding theorem *label*:

\begin{align*}
    \Pr[Y_{t_0} \geq b - \varepsilon t_0] \leq t_0 D p e^{-\gamma(b - \varepsilon t_0 - a)}
\end{align*}

Lets rewrite some to $T_x$: 

\begin{align*}
    \Pr[T_x \leq t_0] \leq t_0 D p e^{-\gamma(b - \varepsilon t_0 - a)}
\end{align*}

Replace $t_0$ with another expression:

\begin{align*}
    t_0 := \frac{\kappa (b - a)}{\varepsilon}
\end{align*}

And then:

\begin{align*}
    \Pr\left[T_x \leq \frac{\kappa (b - a)}{\varepsilon}\right] \leq \frac{\kappa (b - a)}{\varepsilon} D p e^{-\gamma(1 - \kappa)(b - a)}
\end{align*}

For example use $\kappa = \frac{1}{2}$:

\begin{align*}
    \Pr\left[T_x > \frac{(b - a)}{2\varepsilon}\right] \leq \frac{(b - a)}{2\varepsilon} D p e^{-\gamma\frac{(b - a)}{2}}
\end{align*}
\subsection{Lower bounds}
Lets take a look on process $\{X_t\}$ and its derivative process $\{Y_t\}$, where $Y_t = \varepsilon t - X_t$ and it has negative drift.

Then we want to bounds $\Pr[T_x > t_0]$, where $T_x$ - first time $X_t \geq b$. 

\begin{align*}
    \Pr[T_x > t_0] \leq \Pr[X_{t_0} < b] \leq \Pr[Y_{t_0} > \varepsilon t_0 - b]
\end{align*}

Then, because $\{Y_t\}$ - process with negative drift, we may use the corresponding theorem *label*:

\begin{align*}
    \Pr[Y_t > \varepsilon t_0 - b] \leq t_0 D p e^{-\gamma(\epsilon t_0 - b + a)}
\end{align*}

Lets rewrite some to $T_x$: 

\begin{align*}
    \Pr[T_x > t_0] \leq t_0 D p e^{-\gamma(\epsilon t_0 - b + a)}
\end{align*}

Replace $t_0$ with another expression:

\begin{align*}
    t_0 := \frac{\kappa (b - a)}{\varepsilon}
\end{align*}

And then:

\begin{align*}
    \Pr\left[T_x > \frac{\kappa (b - a)}{\varepsilon}\right] \leq \frac{\kappa (b - a)}{\varepsilon} D p e^{-\gamma(\kappa - 1)(b - a)}
\end{align*}

For example use $\kappa = \frac{3}{2}$:

\begin{align*}
    \Pr\left[T_x > \frac{3 (b - a)}{2\varepsilon}\right] \leq \frac{3(b - a)}{2\varepsilon} D p e^{-\gamma\frac{(b - a)}{2}}
\end{align*}

\section{Time bounds for process with variable $\lambda$}
\cm
\subsection{Case when $\lambda$ is random value}
Lets calculate total progress $\triangle X$. It is summation of all progresses of each iterations $\triangle X_i$. Then we can define variable $T$ - number of iterations (or time) and also:
\[
    \triangle X = \sum_{i = 0}^{T} \triangle X_i
\]

Then drift of $\triangle X$ is drift of this Vald's equality: 

\[
    E[\triangle X] = E\left[\sum_{i = 1}^{T} \triangle X_i\right]      
\]

Lets denote that $E[\triangle X_i]$ is a constant or we can show lower and upper bounds. Then:
\[
    E[\triangle X] = E\left[\sum_{i = 1}^{T} \triangle X_i\right] \thickapprox E[T] E[\triangle X_i]      
\]

Our goal to estimate total progress ($\Lambda$) by total cost, which is total number of calculations (number of calculations on iteration $i$ equals $\lambda_i$). Because $\lambda$ is random variable, each $\lambda_i$ has same drift and then:
\[
    E\left[\Lambda\right] = E\left[\sum_{i = 1}^{T} \lambda_i\right] = E[T] E[\lambda]    
\]

And then: 
\[
        E[\Lambda] = \frac{E[\triangle X] E[\lambda]}{E[\triangle X_i]}
\]

Lets denote $\vartheta_\lambda = \frac{E[X_i]}{E[\lambda]}$ and simplify previous equation:
\[
        E[\Lambda] = \frac{E[\triangle X]}{\vartheta_\lambda}
\]

\section{Conclusion}
\cm
\end{document}